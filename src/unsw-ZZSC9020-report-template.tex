\documentclass[mstat,12pt]{unswthesis}

\usepackage{color}
\usepackage{fancyvrb}
\newcommand{\VerbBar}{|}
\newcommand{\VERB}{\Verb[commandchars=\\\{\}]}
\DefineVerbatimEnvironment{Highlighting}{Verbatim}{commandchars=\\\{\}}
% Add ',fontsize=\small' for more characters per line
\usepackage{framed}
\definecolor{shadecolor}{RGB}{248,248,248}
\newenvironment{Shaded}{\begin{snugshade}}{\end{snugshade}}
\newcommand{\AlertTok}[1]{\textcolor[rgb]{0.94,0.16,0.16}{#1}}
\newcommand{\AnnotationTok}[1]{\textcolor[rgb]{0.56,0.35,0.01}{\textbf{\textit{#1}}}}
\newcommand{\AttributeTok}[1]{\textcolor[rgb]{0.77,0.63,0.00}{#1}}
\newcommand{\BaseNTok}[1]{\textcolor[rgb]{0.00,0.00,0.81}{#1}}
\newcommand{\BuiltInTok}[1]{#1}
\newcommand{\CharTok}[1]{\textcolor[rgb]{0.31,0.60,0.02}{#1}}
\newcommand{\CommentTok}[1]{\textcolor[rgb]{0.56,0.35,0.01}{\textit{#1}}}
\newcommand{\CommentVarTok}[1]{\textcolor[rgb]{0.56,0.35,0.01}{\textbf{\textit{#1}}}}
\newcommand{\ConstantTok}[1]{\textcolor[rgb]{0.00,0.00,0.00}{#1}}
\newcommand{\ControlFlowTok}[1]{\textcolor[rgb]{0.13,0.29,0.53}{\textbf{#1}}}
\newcommand{\DataTypeTok}[1]{\textcolor[rgb]{0.13,0.29,0.53}{#1}}
\newcommand{\DecValTok}[1]{\textcolor[rgb]{0.00,0.00,0.81}{#1}}
\newcommand{\DocumentationTok}[1]{\textcolor[rgb]{0.56,0.35,0.01}{\textbf{\textit{#1}}}}
\newcommand{\ErrorTok}[1]{\textcolor[rgb]{0.64,0.00,0.00}{\textbf{#1}}}
\newcommand{\ExtensionTok}[1]{#1}
\newcommand{\FloatTok}[1]{\textcolor[rgb]{0.00,0.00,0.81}{#1}}
\newcommand{\FunctionTok}[1]{\textcolor[rgb]{0.00,0.00,0.00}{#1}}
\newcommand{\ImportTok}[1]{#1}
\newcommand{\InformationTok}[1]{\textcolor[rgb]{0.56,0.35,0.01}{\textbf{\textit{#1}}}}
\newcommand{\KeywordTok}[1]{\textcolor[rgb]{0.13,0.29,0.53}{\textbf{#1}}}
\newcommand{\NormalTok}[1]{#1}
\newcommand{\OperatorTok}[1]{\textcolor[rgb]{0.81,0.36,0.00}{\textbf{#1}}}
\newcommand{\OtherTok}[1]{\textcolor[rgb]{0.56,0.35,0.01}{#1}}
\newcommand{\PreprocessorTok}[1]{\textcolor[rgb]{0.56,0.35,0.01}{\textit{#1}}}
\newcommand{\RegionMarkerTok}[1]{#1}
\newcommand{\SpecialCharTok}[1]{\textcolor[rgb]{0.00,0.00,0.00}{#1}}
\newcommand{\SpecialStringTok}[1]{\textcolor[rgb]{0.31,0.60,0.02}{#1}}
\newcommand{\StringTok}[1]{\textcolor[rgb]{0.31,0.60,0.02}{#1}}
\newcommand{\VariableTok}[1]{\textcolor[rgb]{0.00,0.00,0.00}{#1}}
\newcommand{\VerbatimStringTok}[1]{\textcolor[rgb]{0.31,0.60,0.02}{#1}}
\newcommand{\WarningTok}[1]{\textcolor[rgb]{0.56,0.35,0.01}{\textbf{\textit{#1}}}}




%%%%%%%%%%%%%%%%%%%%%%%%%%%%%%%%%%%%%%%%%%%%%%%%%%%%%%%%%%%%%%%%%%
% 
% OK...Now we get to some actual input.  The first part sets up
% the title etc that will appear on the front page
%
%%%%%%%%%%%%%%%%%%%%%%%%%%%%%%%%%%%%%%%%%%%%%%%%%%%%%%%%%%%%%%%%%

\title{Capstone Project by Team 22\\[0.5cm]A Data Science Approach to
Forecast Electricity Consumption in Australia}

\authornameonly{John Student (z123456), Jim Student2 (zID), Jack
Student3 (zID). }

\author{\Authornameonly}

\copyrightfalse
\figurespagefalse
\tablespagefalse

%%%%%%%%%%%%%%%%%%%%%%%%%%%%%%%%%%%%%%%%%%%%%%%%%%%%%%%%%%%%%%%%%
%
%  And now the document begins
%  The \beforepreface and \afterpreface commands puts the
%  contents page etc in
%
%%%%%%%%%%%%%%%%%%%%%%%%%%%%%%%%%%%%%%%%%%%%%%%%%%%%%%%%%%%%%%%%%%


%%%%%%%%%%%%%%%%%%%%%%%%%%%%%%%%%%%%%%%%%%%%%%%%%%%%%%%%%%%%%%%%%%%%%%%
%
%  A small sample UNSW Coursework Masters thesis file.
%  Any questions to Ian Doust i.doust@unsw.edu.au and/or Gery Geenens ggeenens@unsw.edu.au
%
%%%%%%%%%%%%%%%%%%%%%%%%%%%%%%%%%%%%%%%%%%%%%%%%%%%%%%%%%%%%%%%%%%%%%%%
%
%  The first part pulls in a UNSW Thesis class file.  This one is
%  slightly nonstandard and has been set up to do a couple of
%  things automatically
%
 
%%%%%%%%%%%%%%%%%
%% Precisely one of the next four lines should be uncommented.
%% Choose the one which matches your degree, uncomment it, and comment out the other two!
%\documentclass[mfin,12pt]{unswthesis}    %%  For Master of Financial Mathematics 
%\documentclass[mmath,12pt]{unswthesis}   %%  For Master of Mathematics
%\documentclass[mstat,12pt]{unswthesis}  %%  For Master of Statistics
%%%%%%%%%%%%%%%%%



\linespread{1}
\usepackage{amsfonts}
\usepackage{amssymb}
\usepackage{amsthm}
\usepackage{latexsym,amsmath}
\usepackage{graphicx}
\usepackage{afterpage}
\usepackage[colorlinks]{hyperref}
 \hypersetup{
     colorlinks=true,
     linkcolor=blue,
     filecolor=blue,
     citecolor= black,      
     urlcolor=cyan,
     }
\usepackage{textcomp}
\usepackage{longtable}
\usepackage{booktabs}
\usepackage{float}

%%%%%%%%%%%%%%%%%%%%%%%%%%%%%%%%%%%%%%%%%%%%%%%%%%%%%%%%%%%%%%%%%
%
%  The following are some simple LaTeX macros to give some
%  commonly used letters in funny fonts. You may need more or less of
%  these
%
\newcommand{\R}{\mathbb{R}}
\newcommand{\Q}{\mathbb{Q}}
\newcommand{\C}{\mathbb{C}}
\newcommand{\N}{\mathbb{N}}
\newcommand{\F}{\mathbb{F}}
\newcommand{\PP}{\mathbb{P}}
\newcommand{\T}{\mathbb{T}}
\newcommand{\Z}{\mathbb{Z}}
\newcommand{\B}{\mathfrak{B}}
\newcommand{\BB}{\mathcal{B}}
\newcommand{\M}{\mathfrak{M}}
\newcommand{\X}{\mathfrak{X}}
\newcommand{\Y}{\mathfrak{Y}}
\newcommand{\CC}{\mathcal{C}}
\newcommand{\E}{\mathbb{E}}
\newcommand{\cP}{\mathcal{P}}
\newcommand{\cS}{\mathcal{S}}
\newcommand{\A}{\mathcal{A}}
\newcommand{\ZZ}{\mathcal{Z}}
%%%%%%%%%%%%%%%%%%%%%%%%%%%%%%%%%%%%%%%%%%%%%%%%%%%%%%%%%%%%%%%%%%%%%
%
% The following are much more esoteric commands that I have left in
% so that this file still processes. Use or delete as you see fit
%
\newcommand{\bv}[1]{\mbox{BV($#1$)}}
\newcommand{\comb}[2]{\left(\!\!\!\begin{array}{c}#1\\#2\end{array}\!\!\!\right)
}
\newcommand{\Lat}{{\rm Lat}}
\newcommand{\var}{\mathop{\rm var}}
\newcommand{\Pt}{{\mathcal P}}
\def\tr(#1){{\rm trace}(#1)}
\def\Exp(#1){{\mathbb E}(#1)}
\def\Exps(#1){{\mathbb E}\sparen(#1)}
\newcommand{\floor}[1]{\left\lfloor #1 \right\rfloor}
\newcommand{\ceil}[1]{\left\lceil #1 \right\rceil}
\newcommand{\hatt}[1]{\widehat #1}
\newcommand{\modeq}[3]{#1 \equiv #2 \,(\text{mod}\, #3)}
\newcommand{\rmod}{\,\mathrm{mod}\,}
\newcommand{\p}{\hphantom{+}}
\newcommand{\vect}[1]{\mbox{\boldmath $ #1 $}}
\newcommand{\reff}[2]{\ref{#1}.\ref{#2}}
\newcommand{\psum}[2]{\sum_{#1}^{#2}\!\!\!'\,\,}
\newcommand{\bin}[2]{\left( \begin{array}{@{}c@{}}
				#1 \\ #2
			\end{array}\right)	}
%
%  Macros - some of these are in plain TeX (gasp!)
%
\newcommand{\be}{($\beta$)}
\newcommand{\eqp}{\mathrel{{=}_p}}
\newcommand{\ltp}{\mathrel{{\prec}_p}}
\newcommand{\lep}{\mathrel{{\preceq}_p}}
\def\brack#1{\left \{ #1 \right \}}
\def\bul{$\bullet$\ }
\def\cl{{\rm cl}}
\let\del=\partial
\def\enditem{\par\smallskip\noindent}
\def\implies{\Rightarrow}
\def\inpr#1,#2{\t \hbox{\langle #1 , #2 \rangle} \t}
\def\ip<#1,#2>{\langle #1,#2 \rangle}
\def\lp{\ell^p}
\def\maxb#1{\max \brack{#1}}
\def\minb#1{\min \brack{#1}}
\def\mod#1{\left \vert #1 \right \vert}
\def\norm#1{\left \Vert #1 \right \Vert}
\def\paren(#1){\left( #1 \right)}
\def\qed{\hfill \hbox{$\Box$} \smallskip}
\def\sbrack#1{\Bigl \{ #1 \Bigr \} }
\def\ssbrack#1{ \{ #1 \} }
\def\smod#1{\Bigl \vert #1 \Bigr \vert}
\def\smmod#1{\bigl \vert #1 \bigr \vert}
\def\ssmod#1{\vert #1 \vert}
\def\sspmod#1{\vert\, #1 \, \vert}
\def\snorm#1{\Bigl \Vert #1 \Bigr \Vert}
\def\ssnorm#1{\Vert #1 \Vert}
\def\sparen(#1){\Bigl ( #1 \Bigr )}

\newcommand\blankpage{%
    \null
    \thispagestyle{empty}%
    \addtocounter{page}{-1}%
    \newpage}

%%%%%%%%%%%%%%%%%%%%%%%%%%%%%%%
%
% These environments allow you to get nice numbered headings
%  for your Theorems, Definitions etc.  
%
%  Environments
%
%%%%%%%%%%%%%%%%%%%%%%%%%%%%%%%

\newtheorem{theorem}{Theorem}[section]
\newtheorem{lemma}[theorem]{Lemma}
\newtheorem{proposition}[theorem]{Proposition}
\newtheorem{corollary}[theorem]{Corollary}
\newtheorem{conjecture}[theorem]{Conjecture}
\newtheorem{definition}[theorem]{Definition}
\newtheorem{example}[theorem]{Example}
\newtheorem{remark}[theorem]{Remark}
\newtheorem{question}[theorem]{Question}
\newtheorem{notation}[theorem]{Notation}
\numberwithin{equation}{section}

%%%%%%%%%%%%%%%%%%%%%%%%%%%%%%%%%%%%%%%%%%%%%%%%%%%%%%%%%%%%%%%%%%
%
%  If you've got some funny special words that LaTeX might not
% hyphenate properly, you can give it a helping hand:
%

\hyphenation{Mar-cin-kie-wicz Rade-macher}






\begin{document}

\beforepreface

%\afterpage{\blankpage}

% plagiarism

\prefacesection{Plagiarism statement}

\vskip 2pc \noindent I declare that this thesis is my
own work, except where acknowledged, and has not been submitted for
academic credit elsewhere. 

\vskip 2pc  \noindent I acknowledge that the assessor of this
thesis may, for the purpose of assessing it:
\begin{itemize}
\item Reproduce it and provide a copy to another member of the University; and/or,
\item Communicate a copy of it to a plagiarism checking service (which may then retain a copy of it on its database for the purpose of future plagiarism checking).
\end{itemize}

\vskip 2pc \noindent I certify that I have read and understood the University Rules in
respect of Student Academic Misconduct, and am aware of any potential plagiarism penalties which may 
apply.\vspace{24pt}

\vskip 2pc \noindent By signing 
this declaration I am
agreeing to the statements and conditions above.
\vskip 2pc \noindent
Signed: \rule{7cm}{0.25pt} \hfill Date: \rule{4cm}{0.25pt} \\[1cm]
Signed: \rule{7cm}{0.25pt} \hfill Date: \rule{4cm}{0.25pt} \\[1cm]
Signed: \rule{7cm}{0.25pt} \hfill Date: \rule{4cm}{0.25pt} \\[1cm]
\vskip 1pc

%\afterpage{\blankpage}

% Acknowledgements are optional


\prefacesection{Acknowledgements}

{\bigskip}By far the greatest thanks must go to my supervisor for the
guidance, care and support they provided.\\[1cm] Thanks must also go to
Emily, Michelle, John and Alex who helped by proof-reading the document
in the final stages of preparation.\\[1cm] Although I have not lived
with them for a number of years, my family also deserve many thanks for
their encouragement. Thanks go to Robert Taggart for allowing his thesis
style to be shamelessly copied.\\[1cm] 

{\bigskip\bigskip\bigskip\noindent} 25/07/2020.

%\afterpage{\blankpage}

% Abstract

\prefacesection{Abstract}

The image below gives you some hint about how to write a good abstract.

\par

\bigskip \includegraphics[width=10cm,height=10cm]{good-abstract.png}

%\afterpage{\blankpage}


\afterpreface





%%%%%%%%%%%%%%%%%%%%%%%%%%%%%%%%%%%%%%%%%%%%%%%%%%%%%%%%%%%%%%%%%%
%
% Now we can start on the first chapter
% Within chapters we have sections, subsections and so forth
%
%%%%%%%%%%%%%%%%%%%%%%%%%%%%%%%%%%%%%%%%%%%%%%%%%%%%%%%%%%%%%%%%%%



%%%%%%%%%%%%%%%%%%%%%%%%%%%%%%%%%%%%%

%\afterpage{\blankpage}


\hypertarget{introduction}{%
\chapter{Introduction}\label{introduction}}

This R Markdown template can be used for the ZZSC9020 course report. You
can incorporate R (R Core Team, 2017) chunks and Python chunks that will
be run on the fly. You can incorporate \LaTeX~commands.

\bigskip

Before submitting the last version of your report, you might want to use
\url{https://overleaf.com} to collaborate with other members of your
team directly on the \LaTeX~version of this document (which is a
byproduct you get when you Knit it from studio).

\bigskip

We suggest you organise your report using the following chapters but,
depending on your own project, nothing prevents you to have a different
organisation.

\hypertarget{literature-review}{%
\chapter{Literature Review}\label{literature-review}}

Here are a few references that can be useful: (Xie et al., 2018) and
(Lafaye de Micheaux et al., 2013). See also
\url{https://bookdown.org/yihui/rmarkdown-cookbook/}

\bigskip

In order to incorporate your own references in this report, we strongly
advise you use BibTeX. Your references then needs to be recorded in the
file \texttt{references.bib}.

\hypertarget{material-and-methods}{%
\chapter{Material and Methods}\label{material-and-methods}}

\hypertarget{software}{%
\section{Software}\label{software}}

R and Python of course are great software for Data Science. Sometimes,
you might want to use \texttt{bash} utilities such as \texttt{awk} or
\texttt{sed}.

Of course, to ensure reproducibility, you should use something like
\texttt{Git} and RMarkdown (or a Jupyter Notebook). Do \textbf{not} use
Word!

\hypertarget{description-of-the-data}{%
\section{Description of the Data}\label{description-of-the-data}}

How are the data stored? What are the sizes of the data files? How many
files? etc.

\hypertarget{pre-processing-steps}{%
\section{Pre-processing Steps}\label{pre-processing-steps}}

What did you have to do to transform the data so that they become
useable?

\hypertarget{data-cleaning}{%
\section{Data Cleaning}\label{data-cleaning}}

How did you deal with missing data? etc.

\hypertarget{assumptions}{%
\section{Assumptions}\label{assumptions}}

What assumptions are you making on the data?

\hypertarget{modelling-methods}{%
\section{Modelling Methods}\label{modelling-methods}}

\hypertarget{exploratory-data-analysis}{%
\chapter{Exploratory Data Analysis}\label{exploratory-data-analysis}}

This is where you explore your data using histograms, scatterplots,
boxplots, numerical summaries, etc.

\hypertarget{using-r}{%
\section{Using R}\label{using-r}}

\begin{Shaded}
\begin{Highlighting}[]
\FunctionTok{boxplot}\NormalTok{(cars, }\AttributeTok{col =} \FunctionTok{c}\NormalTok{(}\StringTok{"\#5975a4"}\NormalTok{, }\StringTok{"\#cc8963"}\NormalTok{))}
\end{Highlighting}
\end{Shaded}

\includegraphics{unsw-ZZSC9020-report-template_files/figure-latex/unnamed-chunk-1-1.pdf}

\hypertarget{using-python}{%
\section{Using Python}\label{using-python}}

See
\url{https://cran.r-project.org/web/packages/reticulate/vignettes/r_markdown.html}
for more details.

\bigskip

You need to install the R package \texttt{reticulate}.

\begin{Shaded}
\begin{Highlighting}[]
\BuiltInTok{print}\NormalTok{(}\StringTok{"Python can be used with MATHxxxx!"}\NormalTok{)}
\end{Highlighting}
\end{Shaded}

\begin{verbatim}
## Python can be used with MATHxxxx!
\end{verbatim}

\begin{Shaded}
\begin{Highlighting}[]
\ImportTok{import}\NormalTok{ sys}
\BuiltInTok{print}\NormalTok{(sys.version)}
\end{Highlighting}
\end{Shaded}

\begin{verbatim}
## 3.6.13 (default, Feb 19 2021, 05:17:09) [MSC v.1916 64 bit (AMD64)]
\end{verbatim}

\begin{Shaded}
\begin{Highlighting}[]
\ImportTok{import}\NormalTok{ numpy }\ImportTok{as}\NormalTok{ np}
\NormalTok{np.random.seed(}\DecValTok{1}\NormalTok{)}
\NormalTok{np.random.normal(}\FloatTok{0.0}\NormalTok{, }\FloatTok{1.0}\NormalTok{, size}\OperatorTok{=}\DecValTok{10}\NormalTok{)}
\end{Highlighting}
\end{Shaded}

\begin{verbatim}
## array([ 1.62434536, -0.61175641, -0.52817175, -1.07296862,  0.86540763,
##        -2.3015387 ,  1.74481176, -0.7612069 ,  0.3190391 , -0.24937038])
\end{verbatim}

\begin{Shaded}
\begin{Highlighting}[]
\ImportTok{import}\NormalTok{ pandas }\ImportTok{as}\NormalTok{ pd}
\ImportTok{import}\NormalTok{ matplotlib.pyplot }\ImportTok{as}\NormalTok{ plt}
\NormalTok{df}\OperatorTok{=}\NormalTok{pd.DataFrame([[}\DecValTok{1}\NormalTok{, }\DecValTok{2}\NormalTok{], [}\DecValTok{3}\NormalTok{, }\DecValTok{4}\NormalTok{], [}\DecValTok{4}\NormalTok{, }\DecValTok{3}\NormalTok{], [}\DecValTok{2}\NormalTok{, }\DecValTok{3}\NormalTok{]])}
\NormalTok{fig }\OperatorTok{=}\NormalTok{ plt.figure(figsize}\OperatorTok{=}\NormalTok{(}\DecValTok{4}\NormalTok{, }\DecValTok{4}\NormalTok{))}
\ControlFlowTok{for}\NormalTok{ i }\KeywordTok{in}\NormalTok{ df.columns:}
\NormalTok{    ax}\OperatorTok{=}\NormalTok{plt.subplot(}\DecValTok{2}\NormalTok{,}\DecValTok{1}\NormalTok{,i}\OperatorTok{+}\DecValTok{1}\NormalTok{) }
\NormalTok{    df[[i]].plot(ax}\OperatorTok{=}\NormalTok{ax)}
    \BuiltInTok{print}\NormalTok{(i)}

\NormalTok{plt.show()}
\end{Highlighting}
\end{Shaded}

\includegraphics{unsw-ZZSC9020-report-template_files/figure-latex/unnamed-chunk-4-1.pdf}

\hypertarget{analysis-and-results}{%
\chapter{Analysis and Results}\label{analysis-and-results}}

\hypertarget{a-first-model}{%
\section{A First Model}\label{a-first-model}}

Having a very simple model is always good so that you can benchmark any
result you would obtain with a more elaborate model.

\bigskip

For example, one can use the linear regression model

\[
Y_i = \beta_0 + \beta_1 x_{1i} + \cdots \beta_p x_{pi} + \epsilon_i, \qquad i=1,\ldots,n.
\] where it is assumed that the \(\epsilon_i\)'s are i.i.d.~\(N(0,1)\).

\hypertarget{discussion}{%
\chapter{Discussion}\label{discussion}}

Put the results you got in the previous chapter in perspective with
respect to the problem studied.

\hypertarget{conclusion-and-further-issues}{%
\chapter{Conclusion and Further
Issues}\label{conclusion-and-further-issues}}

What are the main conclusions? What are your recommendations for the
``client''? What further analysis could be done in the future?

A figure:

\begin{figure}
\centering
\includegraphics[width=6cm,height=2cm]{unsw-logo.png}
\caption{A caption \label{myfigure}}
\end{figure}

In the text, see Figure \ref{myfigure}.

\hypertarget{references}{%
\chapter*{References}\label{references}}
\addcontentsline{toc}{chapter}{References}

\hypertarget{refs}{}
\begin{CSLReferences}{0}{0}
\leavevmode\vadjust pre{\hypertarget{ref-Lafaye2013}{}}%
Lafaye de Micheaux, P., Drouilhet, R. and Liquet, B. (2013) \emph{The
{R} software: Fundamentals of programming and statistical analysis}.
Springer New York. Available at:
\url{https://books.google.fr/books?id=Ji-8BAAAQBAJ}.

\leavevmode\vadjust pre{\hypertarget{ref-R}{}}%
R Core Team (2017) \emph{{R}: A language and environment for statistical
computing}. Vienna, Austria: R Foundation for Statistical Computing.
Available at: \url{https://www.R-project.org/}.

\leavevmode\vadjust pre{\hypertarget{ref-Xie2018}{}}%
Xie, Y., Allaire, J.J. and Grolemund, G. (2018) \emph{R markdown, the
definitive guide}. Chapman; Hall/CRC. Available at:
\url{https://bookdown.org/yihui/rmarkdown/}.

\end{CSLReferences}

\bibliographystyle{elsarticle-harv}
\bibliography{references}

\hypertarget{appendix}{%
\chapter*{Appendix}\label{appendix}}
\addcontentsline{toc}{chapter}{Appendix}

\hypertarget{codes}{%
\section*{\texorpdfstring{\textbf{Codes}}{Codes}}\label{codes}}
\addcontentsline{toc}{section}{\textbf{Codes}}

Add you codes here.

\hypertarget{tables}{%
\section*{\texorpdfstring{\textbf{Tables}}{Tables}}\label{tables}}
\addcontentsline{toc}{section}{\textbf{Tables}}

If you have tables, you can add them here.

Use \url{https://www.tablesgenerator.com/markdown_tables} to crete very
simple markdown tables, otherwise use \LaTeX.

\begin{longtable}[]{@{}lcr@{}}
\toprule
Tables & Are & Cool \\
\midrule
\endhead
col 1 is & left-aligned & \$1600 \\
col 2 is & centered & \$12 \\
col 3 is & right-aligned & \$1 \\
\bottomrule
\end{longtable}







\end{document}

